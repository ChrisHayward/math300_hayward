\documentclass[12pt]{article}
\usepackage{fullpage}
\usepackage{amsmath}
\usepackage{amsthm}
\newtheorem{theorem}{Theorem}
\author{Chris Hayward}
\begin{document}
\section*{The Lorentz Condition:}

\begin{eqnarray}
\frac{1}{c} \frac{\partial \phi}{\partial t} + \mbox{div}(A) & = & 0.
\end{eqnarray}
As we shall see, by using what are known as gauge transformations, we can always select potentials for the electromagnetic field that satisfy this condition. The nice part about having the potentials satisfy the Lorentz condition is that the PDEs (9.51-9.52) decouple into a pair of wave equations:
\begin{align*}
\frac{\partial^2 \phi}{\partial t^2 } -c^2\nabla^2 \phi & = 4\pi c^2 \rho, \\
\frac{\partial^2 A}{\partial t^2} -c^2 \nabla^2 A & = 4\pi cJ.
\end{align*}

\begin{theorem}[Lorentz Potential Equations]
On a simply connected spatial region, the vector fields E, B are a solution of Maxwell's equations if and only if
\begin{eqnarray}
E &=& -\nabla \phi - \frac{1}{c} \frac{\partial A}{\partial t},\label{eq1} \\
B &=& \emph{curl}(A),\label{eq2}
\end{eqnarray}

for some scalar field $\phi$ and a vector field A that satisfy the Lorentz potential equations

\begin{eqnarray}
\frac{1}{c} \frac{\partial \phi}{\partial t} + \emph{div}(A) &=& 0, \\
\frac{\partial^2 \phi}{\partial t^2} - c^2 \nabla^2 \phi &=& 4\pi c^2 \rho \\
\frac{\partial^2 A}{\partial t^2} -c^2 \nabla^2 A &=& 4\pi cJ.
\end{eqnarray}

\end{theorem}
\begin{proof}[\emph{\textbf{Proof}}]
Suppose first that E, B is a solution of Maxwell's equations. We repeat some of the above arguments because we have to change the notation slightly. You will see why shortly. Thus, since $\mbox{div}(B) = 0$, the exists a vector field $A_0$ such that $\mbox{curl}(A_0) = B$. Substituting this expression for B into Faraday's law gives $\mbox{curl}(\partial A_0/\partial t + E) = 0$. Thus there exists a scalar field $\phi_0$ such that $-\nabla\phi_0 = \partial A_0/\partial t + E$. Rearranging this give $E = -\nabla\phi_0 - \partial A_0/\partial t$. Thus E and B are given by potentials $\phi_0$ and $A_0$ in the form of equations (\ref{eq1})-(\ref{eq2}).\phantom\qedhere
\end{proof}
\end{document}
